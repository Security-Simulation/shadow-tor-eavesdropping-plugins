\begin{frame}[noframenumbering]
	\frametitle{Diffie-Hellman}
	\begin{itemize}
		\item Alice pick a random number $a$, a prime number $p$ and
		$\alpha$ as a primitive root of p.
		\item Alice calculate $k_a = \alpha ^{a} mod p$ and sends $<k_a,
		p, \alpha>$ over the channel.
		\item Bob read the packet, pick a random number $b$,
		calculate $k_b = \alpha^{b} mod p$ and sends it to Alice.
	\end{itemize}
	Now the shared key $K = k_b^{a} = k_a^{b} = \alpha^{ab} mod p$ is known
	to Alice and Bob\footnote{For the little Fermat theorem ($a^p \equiv a$
	$mod p$) if p is a prime}.
\end{frame}

\begin{frame}[noframenumbering]
	\frametitle{Perfect Forward Secrecy}
	\begin{itemize}
		\item If a key is derived from another with a deterministic
		method then a leak of the second key can reveal every
		eavesdropped transmission encrypthed with the first key.
		\item The immunity to this kind of attacks is called
		\textit{Perfect Forward Security}.
		\item Used in Diffie Hellman based TLS, OTR, etc.
	\end{itemize}
\end{frame}

\begin{frame}[noframenumbering]
	\frametitle{Freenet}
\end{frame}
