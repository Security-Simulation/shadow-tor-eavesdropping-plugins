\section{Conclusions}
We presented in this document an introduction to the Tor network and to
the time analysis based attack. We also built a set of tools in order
to test the attack feasibility. 
\\\\
Furthermore we pointed out that even if the time analysis logic may be simple there are many critical
aspects to be considered and many parameters with which an attacker can
play in order to retrieve as much information as possible. As expected
and as our
results seem to confirm, this kind of attack is feasible but a large
amount of resources, in terms of tracer nodes and analyzer machines, 
is needed in order to obtain a reasonable quantity of data that might be useful to
understand the relations about the Tor network end-points. 
\\\\
It is not to
be ruled out that attackers with such requisites may exist and that may be
interested in such time analysis based attacks. For example there are recent
articles\cite{vantor}\cite{schneier2013attacking} that underlines how
governmental agencies may be interested in
these kind of attacks against Tor.  
\\\\
This kind of vulnerabilities seems to be deeply-rooted in the Tor network architecture.
This can be a big problem for everyone that wants to be anonymous on the internet if
some powerful agencies (or some powerful malicious attacker) will decide to
silently deface Tor based anonymity.
\\\\
In any case, Tor network vulnerabilities are largely considered in many
scientific researches and studies about possible solutions are still
rising and causing worldwide interest.

