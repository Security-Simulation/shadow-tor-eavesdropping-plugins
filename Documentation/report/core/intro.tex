\section{Introduction}
Protecting data privacy on the Web is a very hot topic nowadays. 
Users of the Web may want to surf without the risk that their personal
informations can be read by other users. %TODO and governal agencies
%that may be also intersted in our data, cite the article about it
One of the most largely-used
architecture for this purpose is the \emph{Onion Routing} and its
protocol implementation: \emph{Tor}\cite{tor}. In fact, the latter is modeled with
several techniques with the aim to provide communication security and
data privacy to its network users. Anyway there have been recent papers
 pointing out the Tor's vulnerabilities. %TODO ref these papers
 As the Tor community itself stages, ``Tor does not provide protection
 against end-to-end timing attacks''\cite{tor-overview}, thus the
 chance for an attacker to eavesdrop a Tor communication traffic and
 discover the users involved in it by a
 timing analysis is a well known vulnerability of Tor. 
 
 In this kind of analysis, in order to
 identify the source of a communication, the attacker should be able to
 trace the outgoing traffic and the incoming traffic from both the
 entering and exiting node of the communication path.
 Clearly a timing analysis is feasible only under a certain amount of
 conditions that are often hard to satisfy. As instance, discovering
 that a generic user $U$ is connecting to a server $S$ over a Tor
 communication may require tracing the traffic of many users in the
 network, as the
 attacker cannot know which users may be interested in connecting with
 $S$. Also, there may be the need of tracing more than just the
 interested server because the attacker can find a better time
 relation between the user $U$ and another server $S'$ than between $U$
 and the interested server $S$, thus the attacker could exclude $U$ to be a possible
 connection source for $S$.
 In the section
 \ref{sec:tor} we will better describe how Tor works and how the
 end-to-end timing attack could be performed.
 
 In order to test the feasibility and the parameters involved in
 a time analysis attack over the Tor network, we set up a simulation
 scenario in which a series of simulation runs have been performed and some interesting
 empirical results have been taken out and analyzed.
 At the end we will point out how the Tor time analysis vulnerability can be
 critical and we will introduce some proposals to enhance Tor with the
 view of preventing this kind of attacks. %TODO cite the proposals (one
 %of them http://www.ohmygodel.com/publications/usersrouted-ccs13.pdf
 
 %TODO Introduce the simulation work, made to understand how
 %much Tor can be vulnerable
