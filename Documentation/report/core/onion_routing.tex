\section{Onion Routing and Tor}
\label{sec:tor}
Tor is an implementation of the onion routing architecture model. The
onion routing consists in a technique that provides anonymous
connections over a computer network\cite{or}.
This property is achieved closing the communication data stream through a
chain of encryption steps. 

The figure \ref{fig:onion} shows how a
message is cyphered before the communication begins. The communication
source, before sending the message, choses a communication path of nodes
which the keys are known to the sender.
Then the source node is able to create a stack of encryption starting
with the key of the last relay node and then continuing backwards with
the keys of the other relay nodes in the chain. In this way every node
in the communication path can decrypt the package and read the next hop
address. After that the final node receives the message, he can send the
response back to the originator of the data stream. In this phase the response
message is encrypted sequentially by each node in the chain. 
%TODO:figure or-communication.png 
\begin{figure}[H]
	\centering
	\includegraphics[scale=0.35]{onion.png}
	\caption{Message encryption layers.}
	\label{fig:onion}
\end{figure}	

\subsection{Tor architecture overview}
%TODO protects you from traffic analysis
%https://www.torproject.org/about/overview.html.en
\subsection{Tor attacks}

\subsection{Timing Attack}
