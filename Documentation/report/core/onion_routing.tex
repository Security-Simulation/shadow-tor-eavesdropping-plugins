\section{Onion Routing and Tor}
\label{sec:tor}
Tor is an implementation of the onion routing architecture model. The
onion routing consists in a technique that provides anonymous
connections over a computer network\cite{goldschlag1999onion}.
This property is achieved closing the communication data stream through a
chain of encryption steps. 

The figure \ref{fig:onion} shows how a
message is cyphered before the communication begins. The communication
source, before sending the message, choses a communication path of nodes
which the keys are known to the sender.
Then the source node is able to create a stack of encryption starting
with the key of the last relay node and then continuing backwards with
the keys of the other relay nodes in the chain. In this way every node
in the communication path can decrypt the package and read the next hop
address. After that the final node receives the message, he can send the
response back to the originator of the data stream. In this phase the response
message is encrypted sequentially by each node in the chain. 
With this method each relay node can gain access to the
previous and the next node addresses only. Anyway the last node of the
Onion Routing path, called exit node, send the message to the end point
as plain text.

\begin{figure}[H]
	\centering
	\includegraphics[scale=0.30]{onion.png}
	\caption{Message encryption layers.}
	\label{fig:onion}
	\includegraphics[scale=0.30]{or-communication.png}
	\caption{A two relay Onion Router communication.\cite{dingledine2004tor}}
\end{figure}	

\subsection{Tor architecture overview}
As introduced before, Tor is born with the aim to allow people to
improve their privacy and security on the Internet. Its architecture is
based on the Onion Routing model and it's widely used by many user over
the world. Users may be interested on using Tor for different purposes
such as avoiding website tracking, communicating securely over Internet
messaging services, or just web surfing with the access on the services
blocked by their local Internet providers.

The idea behind Tor, and Onion Routing as well, is to protect people
against a common form of Internet surveillance known as "traffic
analysis". Traffic analysis can reveal information about the network
traffic such as the source, destination, size, timing and more of the
analyzed traffic packets. This can be possible even if the packets are
cyphered because the traffic analysis focuses on the header part of the
packets that are in plain text. 

Thus, simply listing between the sender and recipient on the network,
a traffic analysis can be performed. Moreover, spying on multiple parts
of the Internet and using some statistical techniques, some attackers can
track the communications patterns of many different organizations and
individuals.
%TODO: put this part before as an introduction and then put the
%details explained before
%https://www.torproject.org/about/overview.html.en

\subsection{Tor attacks}

\subsection{Timing Attack}
