\section{Simulation}
\label{sec:simulation}
In order to analyze the effectiveness of a time analysis attack in a
realistic scenario, we decide to consider a simulation environment.
We wanted to model the scenario in a way that most of the
characteristics of the Tor network architecture would have been
considered.

The \textbf{shadow simulator}\cite{shadowoverview} seemed to be a good candidate for our purpose. In
the next paragraphs we will introduce shadow and we will illustrate our
work based on this simulator.
 
\subsection{Shadow}
Most of the simulators used for networks experimentation do not provide
the use of third-party already existent applications, which their
behaviour are often simulated as well. 
This approach can be reasonable if the analysis is focused on the 
network layers below the application layer. 

In our case we are interested in studying the characteristic of the Tor
network and the Tor application installed inside the Tor network nodes.
The latter is one of the core part of the Tor architecture as it implements the
onion routing protocol that let the Tor nodes communicate. More
specifically the Tor application daemons communicate each others on the 
top of the current internet network layers (TCP/IP). Above the internet
network layers an ulterior network stack is used by the standard web applications 
(e.g. browser, web servers, etc.) whose communications packets are
tunneled through the Tor network layers by the Tor applications.

It is evident that the Tor application has a significant role in
the Tor architecture, thus we chose the shadow simulator as it permits
the use of the real Tor applications in a simulated environment.
Essentially it allows the execution of a set of local applications that
communicates in a simulated computer network.

Applications are executed inside the shadow simulator through 
dynamic libraries called plugins. 
These plugins are interfaces used by shadow to trace a
selective set of system calls and re-route them to the simulator instead
of letting them proceed directly to the kernel. 
In this way, the simulator is transparent to the application
that may function as like it was running in a standard UNIX environment. 

%TODO: introduce scallion and the other plugins and find some pictures

\subsection{Autosys plugin}
\subsection{Analyzer plugin}
\subsection{Simpletcp plugin}
