\section{Simulation}
\label{sec:simulation}
In order to analyze the effectiveness of a time analysis attack in a
realistic scenario, we decide to consider a simulation environment.
We wanted to model the scenario in a way that most of the
characteristics of the Tor network architecture would have been
considered.

The \textbf{shadow simulator}\cite{shadow} seemed to be a good candidate for our purpose. In
the next paragraphs we will introduce shadow and we will illustrate our
work based on this simulator.
 
\subsection{Shadow}
Most of the simulators used for networks experimentation do not provide
the use of real external applications, which their
behaviour are often simulated as well. 
This approach can be reasonable if the analysis is focused on the 
network layers below the application layer. 

In our case we are interested in studying the characteristic of the Tor
network which is based on the network application layer.
Thus the Tor application is one of the core part of the Tor architecture as it implements the
onion routing protocol that let the Tor nodes communicate each others.

It is evident that the Tor application has a significant role in
the Tor architecture, thus we chose the shadow simulator as it permits
the use of the real Tor applications in a simulated environment.
Essentially it allows the execution of a set of local applications that
communicates in a simulated computer network. Moreover we could
avoid the oversimplification of the system that could have been occurred
with a custom implementation.

Applications are executed inside the shadow simulator through 
dynamic libraries called plug-ins. 
These plug-ins are interfaces used by shadow to trace a
selective set of system functions and re-route them to the simulator instead
of letting them proceed directly to the kernel. 
In this way, the simulator is transparent to the application
that may function as like it was running in a standard UNIX environment. 

The plug-in that allows the execution of the Tor applications is
scallion. The latter provides also some useful tools for the virtual
Tor network topology generation.

The virtual network is modeled with a XML configuration
file, called blueprint, used by shadow to understand the network
structure and some network properties such as link latency, jitter and
packet loss rate. The blueprint also tells Shadow what software each
virtual node should run at its creation. This is specified with a
plug-ins list for each virtual node entry of the XML file.

We implemented three shadow plug-ins and their relative applications 
 able to work alongside scallion in
order to experiment a time analysis attack scenario. Our plug-ins and
the attack scenario will be illustrated in the next paragraphs.

%TODO: find some pictures
\subsection{Simulation Scenario}
%TODO: un bello schema con i nodi e i relativi plug-in

\subsection{Autosys plug-in}
To analyze the tor network traffic we need, at least, some information about
the incoming and the outgoing connections.\\
To achieve this result we have more than a solution: in the first one we can place 
in the last network element that is under the control of an autonomous system\footnote{
That the reason why the plug-in is called "Autosys".} a connection
listener which traces a whole subset of
network nodes;
otherwise we can place the listener in a, malware-like, invisible proxy on the target
machine with the aim to trace every outgoing connection. 
In both solutions we need to place some correspondent exit tracers on
the target end points.
This proxies can be implemented as a simple raw-socket based sniffer. 
Unfortunately Shadow doesn't support raw-socket
% XXX QUA MAGARI CI METTIAMO LA FONTE
emulation, so we implemented the plug-in as a simple event based TCP
proxy. In our attack scenario an instance of the autosys plug-in is placed 
between the client application and the client Tor daemon, also a second
 instance is placed between the server Tor daemon and the server application.
\\
% Qua possiamo mettere un'immagine che spiega il tutto
% scrivi algoritmo

% XXX FORSE DA SPOSTARE IN ANALYZER
When some connection initiator packets flow trough the proxy, the
current machine time of day is sent to the analysis server
% DA CONTROLLARE FOOTNOTE
\footnote{The simulator have a single clock, so we don't have any problem with the
clock synchronization.}.
% Altra immagine

\subsection{Analyzer plug-in}
Data sniffed by the autosys plug-in need to be stored for later analysis, 
to do so we need another plug-in.
This plug-in is called the "Analyzer plug-in" or the "Analysis server"
which will collect all the dead-drops generated by the sniffers.
The plug-in is implemented as a simple event driven UDP server, every
packet received from the sniffer plug-in is so saved in a single place.
% TODO show the packet structure 
% footnote con spiegazione perdita pacchetti 
% dire che sta roba viene analizzata da pynalyzer
% immagine con il salvataggio

\subsection{Simpletcp plug-in}
% spiegare socks
