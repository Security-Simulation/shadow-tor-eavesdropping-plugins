\section{Simulation}
\label{sec:simulation}
In order to analyze the effectiveness of a time analysis attack in a
realistic scenario, we decide to consider a simulation environment.
We wanted to model the scenario in a way that most of the
characteristics of the Tor network architecture would have been
considered.

The \textbf{shadow simulator}\cite{shadow} seemed to be a good candidate for our purpose. In
the next paragraphs we will introduce shadow and we will illustrate our
work based on this simulator.
 
\subsection{Shadow}
Most of the simulators used for networks experimentation do not provide
the use of real external applications, which their
behaviour are often simulated as well. 
This approach can be reasonable if the analysis is focused on the 
network layers below the application layer. 

In our case we are interested in studying the characteristic of the Tor
network and the Tor application installed inside the Tor network nodes.
The latter is one of the core part of the Tor architecture as it implements the
onion routing protocol that let the Tor nodes communicate. More
specifically the Tor application daemons communicate each others on the 
top of the current internet network layers (TCP/IP). Above the internet
network layers an ulterior network stack is used by the standard web applications 
(e.g. browser, web servers, etc.) whose communications packets are
tunneled through the Tor network layers by the Tor applications.

It is evident that the Tor application has a significant role in
the Tor architecture, thus we chose the shadow simulator as it permits
the use of the real Tor applications in a simulated environment.
Essentially it allows the execution of a set of local applications that
communicates in a simulated computer network.

Applications are executed inside the shadow simulator through 
dynamic libraries called plug-ins. 
These plug-ins are interfaces used by shadow to trace a
selective set of system calls and re-route them to the simulator instead
of letting them proceed directly to the kernel. 
In this way, the simulator is transparent to the application
that may function as like it was running in a standard UNIX environment. 

The plug-in that allows the execution of the Tor applications is
scallion. The latter provides also some useful tools for the virtual
Tor network topology generation.

The virtual network is modeled with a XML configuration
file, called blueprint, used by shadow to understand the network
structure and some network properties such as link latency, jitter and
packet loss rate. The blueprint also tells Shadow what software each
virtual node should run at its creation. This is specified with a
plug-ins list for each virtual node entry of the XML file.

We implemented three shadow plug-ins and their relative applications 
 able to work alongside scallion in
order to experiment a time analysis attack scenario. Our plug-ins and
the attack scenario will be illustrated in the next paragraphs.

%TODO: find some pictures
\subsection{Simulation Scenario}
%TODO: un bello schema con i nodi e i relativi plug-in

\subsection{Autosys plugin}
To analyze the tor network traffic we need, at least, some informations about
the incoming and the outgoing connections.\\
To achieve that result we have more than a solution, we can place a connection
listener in the last network element under the control of an autonomous system\footnote{
That the reason why the plugin is called "Autosys".} tracing a whole subset of
network nodes.
Or we can place the listener in a, malware-like, invisible proxy on the target
machine tracing every outgoing connection, we also need to place an exit tracer on the other end.
This proxy can be implemented in some different ways, one of the bests solutions can be
a simple raw-socket based sniffer, unfortunately Shadow doesn't support raw-socket
% XXX QUA MAGARI CI METTIAMO LA FONTE
emulation, so the plugin was implemented as a simple event based TCP proxy interposed
between the client/server application and the Tor entry/exit guard.
\\
% Qua possiamo mettere un'immagine che spiega il tutto

% XXX FORSE DA SPOSTARE IN ANALYZER
When some connection initiator packets flows trough the proxy the current time
is sent to the analysis server
% DA CONTROLLARE FOOTNOTE
\footnote{The simulator have a single clock, so we don't have problem with the
clock syncronization.}.
% Altra immagine

\subsection{Analyzer plugin}
The data sniffed by the autosys plugin need to be stored in a place to be
analyzed later, to do so we need another plugin.
This plugin is called the "Analyzer plugin" or the "Analysis server" this server
will so act to collect all the deadrops generated by the sniffers.
The plugin is implemented as a simple event driven UDP server, every packet
received from the sniffer plugin is so saved in a single place.
% immagine con il salvataggio

\subsection{Simpletcp plugin}
