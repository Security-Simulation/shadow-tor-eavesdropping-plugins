\section{Future Works}
As we stated in the previous sections, this study only cover a little part of
the big field of the timing-analysis attacks\footnote{Or, more in general, side
channel attacks; There is a specific field in cryptanalysis that study this kind
of attacks.} so this work haven't covered some interesting points and study that
can refine or give some more efficient countermeasures to the problem.
\\\\
One of the interesting future works is the test and a benchmark
of the so-called \emph{Astoria}\cite{starov2015measuring} Tor client.
That client rethink
the path-selection algorithm of the standard tor clients to avoid the kind of
attacks we described here, that algorithm operates on an upper level that choose
the routers minimizing the risk that two nodes falls under the realm of the same
autonomous system.
\\\\
Another future work is the comparison of the impact of the attack described above
between the Tor network and the other anonymous network implementations like i2p or freenet.
\\\\
A really intresting work is the study of the, old but currently really inspirative,
Mix Network model\cite{chaum1981untraceable}: the spiritual predecessor
of the onion routing model, this network model is immune to this kind of
attack, because the packets will be transmitted and mixed in random ways,
this can be a starting point to make an efficient non-low latency onion routing
network. Or, hopefully, a powerful mix of the two implementations.
\\\\
An implementation work related to our model can be the modification
of the simpletcp plug-in to make it capable of connecting and download from
multiple servers in a single instance, and the changes
required to the analysis method/algorithm to trace that kinds of connections.
\\\\
As a side note, this experiments can be repeated with a bigger tor
network, but, as what transpire from the paper\emph{AS-awareness in Tor
path selection}\cite{edman2009awareness} this problem is not related to
the proportion of the network.
