\documentclass[a4paper]{article}
\usepackage[english]{babel}
\usepackage[utf8]{inputenc}
\usepackage{hyperref}
\usepackage{graphicx}
\usepackage{wrapfig}
\usepackage{listings}
\usepackage{float}
\usepackage{bytefield}
%\usepackage{fullpage}
\usepackage{algorithm}
\usepackage{algpseudocode}
\usepackage{amsmath}
%\usepackage[colorinlistoftodos]{todonotes}

\title{Eavesdropping analysis over TOR networks}

\author{Matteo Martelli, Davide Berardi}

\date{\today}

\begin{document}
\maketitle

\begin{abstract}
In this document we talk about blablabla
\end{abstract}

\section{Introduction}
Protecting data privacy on the Web is a very hot topic nowadays. 
Users of the Web may want to surf without the risk that their personal
informations can be read by other users. One of the most largely-used
architecture for this purpose is the \emph{Onion Routing} and its
protocol implementation: \emph{Tor}\cite{tor}. In fact, the latter is modeled with
several techniques with the aim to provide communication security and
data privacy to its network users. Anyway there have been recent papers
 pointing out the Tor's vulnerabilities. %TODO ref these papers
%TODO introduce time analyis and that also tor authors admit the
%vulnerability. Introduce the simulation work, made to understand how
%much Tor can be vulnerable

\section{Onion Routing and Tor}

\subsection{Tor architecture overview}

\subsection{Tor attacks}

\subsection{Eavesdropping}

\section{Simulation}
\subsection{Shadow}
\subsection{Autosys plugin}
\subsection{Analyzer plugin}
\subsection{Simpletcp plugin}

\section{Data Analysis}
\subsection{Netbuilder Script}
\subsection{Analyzer Script}
\subsection{Empirical Results}

\section{Conclusions}
%TODO: diciamo che effettivamente sembra molto vulnerabile a time
%analysis anche se  si dovrebbe provare con reti più grandi anche se "AS-awareness in Tor
%path selection" dice che non cambia molto con la crescita della rete.
%Loro propongono anche un algoritmo di selezione dei percorsi che
%dovrebbe andar "meglio". Diamiogli un occhiata e in caso menzioniamolo. 

\begin{thebibliography}{50}
	\bibitem{tor}
		R. Dingledine, N. Mathewson, and P. Syverson. 
		Tor: the second-generation onion router. 
		\textsl:{Proceedings of the 13th conference on USENIX Security Symposium}, 
		pages 21–21, Berkeley, CA, USA, 2004.
		USENIX Association.
\end{thebibliography}
\end{document}
